% \LaTeX-Main\
% !TeX encoding=UTF-8
%% The LaTeX package csvsimple - version 2.6.0pre3 (2023/12/19)
%% csvsimple.tex: Manual
%%
%% -------------------------------------------------------------------------------------------
%% Copyright (c) 2008-2023 by Prof. Dr. Dr. Thomas F. Sturm <thomas dot sturm at unibw dot de>
%% -------------------------------------------------------------------------------------------
%%
%% This work may be distributed and/or modified under the
%% conditions of the LaTeX Project Public License, either version 1.3
%% of this license or (at your option) any later version.
%% The latest version of this license is in
%%   http://www.latex-project.org/lppl.txt
%% and version 1.3 or later is part of all distributions of LaTeX
%% version 2005/12/01 or later.
%%
%% This work has the LPPL maintenance status `author-maintained'.
%%
%% This work consists of all files listed in README.md
%%
\documentclass[a4paper,11pt]{ltxdoc}
\usepackage{csvsimple-doc}

\usepackage{\csvpkgprefix csvsimple-legacy}

\tcbmakedocSubKey{docCsvKey}{csv}

\hypersetup{
  pdftitle={Manual for the csvsimple-legacy package},
  pdfauthor={Thomas F. Sturm},
  pdfsubject={csv file processing with LaTeX2e},
  pdfkeywords={csv file, comma separated values, key value syntax}
}

%%%%%%%%%%%%%%%%%%%%%%%%%%%%%%%%%%%%%%%%%%%%%%%%%
\begin{document}
\begin{center}
\begin{tcolorbox}[enhanced,hbox,tikznode,left=8mm,right=8mm,boxrule=0.4pt,
  colback=white,colframe=black!50!yellow,
  drop lifted shadow=black!50!yellow,arc is angular,
  before=\par\vspace*{5mm},after=\par\bigskip]
{\bfseries\LARGE The \texttt{csvsimple-legacy} package}\\[3mm]
{\large Manual for version \version\ (\datum)}
\end{tcolorbox}
{\large Thomas F.~Sturm%
  \footnote{Prof.~Dr.~Dr.~Thomas F.~Sturm, Institut f\"{u}r Mathematik und Informatik,
    University of the Bundeswehr Munich, D-85577 Neubiberg, Germany;
     email: \href{mailto:thomas.sturm@unibw.de}{thomas.sturm@unibw.de}}\par\medskip
\normalsize\url{https://www.ctan.org/pkg/csvsimple}\par
\url{https://github.com/T-F-S/csvsimple}
}
\end{center}
\bigskip
\begin{absquote}
  \begin{center}\bfseries Abstract\end{center}
  |csvsimple(-legacy)| provides a simple \LaTeX\ interface for the processing of files with
  comma separated values (CSV). |csvsimple-legacy| relies heavily on the key value
  syntax from |pgfkeys| which results in an easy way of usage.
  Filtering and table generation is especially supported. Since the package
  is considered as a lightweight tool, there is no support for data sorting
  or data base storage.
\end{absquote}


\begin{tcolorbox}[enhanced,left=8mm,right=8mm,boxrule=2pt,boxsep=3mm,
  colback=red!85!gray!5!white,colframe=red!85!gray,
  arc is angular,arc=5mm,
  before skip=1cm]
Actually, |csvsimple-legacy| is identical to the old version 1.22 (2021/06/07)
of |csvsimple|. It is superseded by |csvsimple-l3|, a \LaTeX3 implementation
of |csvsimple| which is a \emph{nearly} drop-in for the erstwhile implementation.
\begin{itemize}
\item If you are a new user or an experienced user of |csvsimple| creating a
  new document, you are encouraged to turn to |csvsimple-l3|, see\\
  \href{csvsimple-l3.pdf}{\flqq The |csvsimple-l3| package\frqq}
\item If you used |csvsimple| before version 2.00 in one or many documents,
  there is \emph{no need} to change anything. Loading |csvsimple|
  without options loads |csvsimple-legacy|.
  |csvsimple-legacy| will be maintained to stay functional as it is for the
  sake of compatibility to old documents.
\item Differences between |csvsimple-legacy| and |csvsimple-l3| are
  discussed in \href{csvsimple.pdf}{\flqq The |csvsimple| package\frqq}.
\end{itemize}
\end{tcolorbox}


\clearpage
\tableofcontents

\clearpage
\section{Introduction}%
The |csvsimple-legacy| package is applied to the processing of
CSV\footnote{CSV file: file with comma separated values.} files.
This processing is controlled by key value assignments according to the
syntax of |pgfkeys|. Sample applications of the package
are tabular lists, serial letters, and charts.

An alternative to |csvsimple-legacy| is the |datatool| package
which provides considerably more functions and allows sorting of data by \LaTeX.
|csvsimple-legacy| has a different approach for the user interface and
is deliberately restricted to some basic functions with fast
processing speed.

Mind the following restrictions:
\begin{itemize}
\item Sorting is not supported directly but can be done
  with external tools, see \Fullref{sec:Sorting}.
\item Values are expected to be comma separated, but the package
  provides support for other separators, see \Fullref{sec:separators}.
\item Values are expected to be either not quoted or quoted with
  curly braces |{}| of \TeX\ groups. Other quotes like doublequotes
  are not supported directly, but can be achieved
  with external tools, see \Fullref{sec:importeddata}.
\item Every data line is expected to contain the same amount of values.
  Unfeasible data lines are silently ignored by default, but this can
  be configured, see \Fullref{sec:consistency}.
\end{itemize}


\subsection{Loading the Package}
The package |csvsimple-legacy| loads the packages
|pgfkeys|,
|etoolbox|,
and |ifthen|.
|csvsimple-legacy| itself is loaded with \emph{one} of the following
alternatives inside the preamble:
\begin{dispListing}
\usepackage{csvsimple}
   % or alternatively (not simultaneously!)
\usepackage[legacy]{csvsimple}
   % or alternatively (not simultaneously!)
\usepackage{csvsimple-legacy}
\end{dispListing}


Not automatically loaded, but used for many examples are the packages
|longtable|
and
|booktabs|.


\clearpage
\subsection{First Steps}
Every line of a processable CSV file has to contain an identical amount of
comma\footnote{See \refKey{/csv/separator} for other separators than comma.} separated values. The curly braces |{}| of \TeX\ groups can be used
to mask a block which may contain commas not to be processed as separators.

The first line of such a CSV file is usually but not necessarily a header line
which contains the identifiers for each column.

%-- file embedded for simplicity --
\begin{tcbverbatimwrite}{grade.csv}
name,givenname,matriculation,gender,grade
Maier,Hans,12345,m,1.0
Huber,Anna,23456,f,2.3
Weißbäck,Werner,34567,m,5.0
Bauer,Maria,19202,f,3.3
\end{tcbverbatimwrite}
%-- end embedded file --

\csvlisting{grade}

\smallskip
The most simple way to display a CSV file in tabular form is the processing
with the \refCom{csvautotabular} command.

\begin{dispExample}
\csvautotabular{grade.csv}
\end{dispExample}


Typically, one would use \refCom{csvreader} instead of |\csvautotabular| to
gain full control over the interpretation of the included data.

In the following example, the entries of the header line are automatically
assigned to \TeX\ macros which may be used deliberately.


\begin{dispExample}
\begin{tabular}{|l|c|}\hline%
\bfseries Person & \bfseries Matr.~No.
\csvreader[head to column names]{grade.csv}{}%
{\\\givenname\ \name & \matriculation}%
  \\\hline
\end{tabular}
\end{dispExample}


\clearpage
|\csvreader| is controlled by a plenty of options. For example, for table
applications line breaks are easily inserted by
\refKey{/csv/late after line}. This defines a macro execution just before
the following line.
Additionally, the assignment of columns to \TeX\ macros is shown in a non automated
way.

\begin{dispExample}
\begin{tabular}{|r|l|c|}\hline%
& Person & Matr.~No.\\\hline\hline
\csvreader[late after line=\\\hline]%
  {grade.csv}{name=\name,givenname=\firstname,matriculation=\matnumber}%
  {\thecsvrow & \firstname~\name & \matnumber}%
\end{tabular}
\end{dispExample}

\smallskip
An even more comfortable and preferrable way to create a table is setting
appropriate option keys. Note, that this gives you the possibility to create a
|pgfkeys| style which contains the whole table creation.

\begin{dispExample}
\csvreader[tabular=|r|l|c|,
    table head=\hline & Person & Matr.~No.\\\hline\hline,
    late after line=\\\hline]%
  {grade.csv}{name=\name,givenname=\firstname,matriculation=\matnumber}%
  {\thecsvrow & \firstname~\name & \matnumber}%
\end{dispExample}

\smallskip
The next example shows such a style definition with the convenience macro
\refCom{csvstyle}. Here, we see again the automated assignment of header
entries to column names by \refKey{/csv/head to column names}.
For this, the header entries have to be without spaces and special characters.
But you can always assign entries to canonical macro names by hand like in the examples
above. Here, we also add a \refKey{/csv/head to column names prefix} to avoid
macro name clashes.

\begin{dispExample}
\csvstyle{myTableStyle}{tabular=|r|l|c|,
    table head=\hline & Person & Matr.~No.\\\hline\hline,
    late after line=\\\hline,
    head to column names,
    head to column names prefix=MY,
    }

\csvreader[myTableStyle]{grade.csv}{}%
  {\thecsvrow & \MYgivenname~\MYname & \MYmatriculation}%
\end{dispExample}


\clearpage
Another way to address columns is to use their roman numbers.
The direct addressing is done by |\csvcoli|, |\csvcolii|, |\csvcoliii|, \ldots:

\begin{dispExample}
\csvreader[tabular=|r|l|c|,
    table head=\hline & Person & Matr.~No.\\\hline\hline,
    late after line=\\\hline]%
  {grade.csv}{}%
  {\thecsvrow & \csvcolii~\csvcoli & \csvcoliii}%
\end{dispExample}

\smallskip
And yet another method to assign macros to columns is to use arabic numbers
for the assignment:

\begin{dispExample}
\csvreader[tabular=|r|l|c|,
    table head=\hline & Person & Matr.~No.\\\hline\hline,
    late after line=\\\hline]%
  {grade.csv}{1=\name,2=\firstname,3=\matnumber}%
  {\thecsvrow & \firstname~\name & \matnumber}%
\end{dispExample}

\smallskip
For recurring applications, the |pgfkeys| syntax allows to create own styles
for a consistent and centralized design. The following example is easily
modified to obtain more or less option settings.

\begin{dispExample}
\csvset{myStudentList/.style={%
  tabular=|r|l|c|,
  table head=\hline & Person & #1\\\hline\hline,
  late after line=\\\hline,
  column names={name=\name,givenname=\firstname}
  }}

\csvreader[myStudentList={Matr.~No.}]{grade.csv}{matriculation=\matnumber}%
{\thecsvrow & \firstname~\name & \matnumber}%
\hfill%
\csvreader[myStudentList={Grade}]{grade.csv}{grade=\grade}%
{\thecsvrow & \firstname~\name & \grade}%
\end{dispExample}


\clearpage
Alternatively, column names can be set by \refCom{csvnames}
and style definitions by \refCom{csvstyle}.
With this, the last example is rewritten as follows:

\begin{dispExample}
\csvnames{myNames}{1=\name,2=\firstname,3=\matnumber,5=\grade}
\csvstyle{myStudentList}{tabular=|r|l|c|,
  table head=\hline & Person & #1\\\hline\hline,
  late after line=\\\hline, myNames}

\csvreader[myStudentList={Matr.~No.}]{grade.csv}{}%
{\thecsvrow & \firstname~\name & \matnumber}%
\hfill%
\csvreader[myStudentList={Grade}]{grade.csv}{}%
{\thecsvrow & \firstname~\name & \grade}%
\end{dispExample}

\smallskip
The data lines of a CSV file can also be filtered. In the following example,
a certificate is printed only for students with grade unequal to 5.0.

\begin{dispExample}
\csvreader[filter not strcmp={\grade}{5.0}]%
  {grade.csv}{1=\name,2=\firstname,3=\matnumber,4=\gender,5=\grade}%
  {\begin{center}\Large\bfseries Certificate in Mathematics\end{center}
  \large\ifcsvstrcmp{\gender}{f}{Ms.}{Mr.}
  \firstname~\name, matriculation number \matnumber, has passed the test
  in mathematics with grade \grade.\par\ldots\par
  }%
\end{dispExample}


\clearpage
\section{Macros for the Processing of CSV Files}\label{sec:makros}%

\begin{docCommand}{csvreader}{\oarg{options}\marg{file name}\marg{assignments}\marg{command list}}
  |\csvreader| reads the file denoted by \meta{file name} line by line.
  Every line of the file has to contain an identical amount of
  comma separated values. The curly braces |{}| of \TeX\ groups can be used
  to mask a block which may contain commas not to be processed as separators.\smallskip

  The first line of such a CSV file is by default but not necessarily
  processed as a header line which contains the identifiers for each column.
  The entries of this line can be used to give \meta{assignments} to \TeX\ macros
  to address the columns. The number of entries of this first line
  determines the accepted number of entries for all following lines.
  Every line which contains a higher or lower number of entries is ignored
  during standard processing.\smallskip

  The \meta{assignments} are given by key value pairs
  \mbox{\meta{name}|=|\meta{macro}}. Here, \meta{name} is an entry from the
  header line \emph{or} the arabic number of the addressed column.
  \meta{macro} is some \TeX\ macro which gets the content of the addressed column.\smallskip

  The \meta{command list} is executed for every accepted data line. Inside the
  \meta{command list} is applicable:
  \begin{itemize}
  \item \docAuxCommand{thecsvrow} or the counter |csvrow| which contains the number of the
    current data line (starting with 1).
  \item \docAuxCommand{csvcoli}, \docAuxCommand{csvcolii}, \docAuxCommand{csvcoliii}, \ldots,
    which contain the contents of the column entries of the current data line.
    Alternatively can be used:
  \item \meta{macro} from the \meta{assignments} to have a logical
    addressing of a column entry.
  \end{itemize}
  Note, that the \meta{command list} is allowed to contain |\par| and
  that all macro definitions are made global to be used for table applications.\smallskip

  The processing of the given CSV file can be controlled by various
  \meta{options} given as key value list. The feasible option keys
  are described in section \ref{sec:schluessel} from page \pageref{sec:schluessel}.

\begin{dispExample}
\csvreader[tabular=|r|l|l|, table head=\hline, table foot=\hline]{grade.csv}%
  {name=\name,givenname=\firstname,grade=\grade}%
  {\grade & \firstname~\name & \csvcoliii}
\end{dispExample}

Mainly, the |\csvreader| command consists of a \refCom{csvloop} macro with
following parameters:\par
|\csvloop{|\meta{options}|, file=|\meta{file name}|, column names=|\meta{assignments}|,|\\
  \hspace*{2cm} |command=|\meta{command list}|}|\par
  Therefore, the application of the keys \refKey{/csv/file} and \refKey{/csv/command}
is useless for |\csvreader|.
\end{docCommand}

\begin{docCommand}{csvloop}{\marg{options}}
  Usually, \refCom{csvreader} may be preferred instead of |\csvloop|.
  \refCom{csvreader} is based on |\csvloop| which takes a mandatory list of
  \meta{options} in key value syntax.
  This list of \meta{options} controls the total processing. Especially,
  it has to contain the CSV file name.
\begin{dispExample}
\csvloop{file={grade.csv}, head to column names, command=\name,
  before reading={List of students:\ },
  late after line={{,}\ }, late after last line=.}
\end{dispExample}
\end{docCommand}

\clearpage
The following |\csvauto...| commands are intended for quick data overview
with limited formatting potential.
See Subsection~\ref{subsec:tabsupport} on page \pageref{subsec:tabsupport}
for the general table options in combination with \refCom{csvreader} and
\refCom{csvloop}.

\begin{docCommand}{csvautotabular}{\oarg{options}\marg{file name}}
  |\csvautotabular| is an abbreviation for the application of the option key
  \refKey{/csv/autotabular} together with other \meta{options} to \refCom{csvloop}.
  This macro reads the whole CSV file denoted by \meta{file name}
  with an automated formatting.
\begin{dispExample}
\csvautotabular{grade.csv}
\end{dispExample}
\begin{dispExample}
\csvautotabular[filter equal={\csvcoliv}{f}]{grade.csv}
\end{dispExample}
\end{docCommand}


\begin{docCommand}{csvautolongtable}{\oarg{options}\marg{file name}}
  |csvautolongtable| is an abbreviation for the application of the option key
  \refKey{/csv/autolongtable} together with other \meta{options} to \refCom{csvloop}.
  This macro reads the whole CSV file denoted by \meta{file name}
  with an automated formatting.
  For application, the package |longtable| is required which has to be
  loaded in the preamble.
\begin{dispListing}
\csvautolongtable{grade.csv}
\end{dispListing}
\csvautolongtable{grade.csv}
\end{docCommand}

\clearpage

\begin{docCommand}{csvautobooktabular}{\oarg{options}\marg{file name}}
  |\csvautobooktabular| is an abbreviation for the application of the option key
  \refKey{/csv/autobooktabular} together with other \meta{options} to \refCom{csvloop}.
  This macro reads the whole CSV file denoted by \meta{file name}
  with an automated formatting.
  For application, the package |booktabs| is required which has to be
  loaded in the preamble.
\begin{dispExample}
\csvautobooktabular{grade.csv}
\end{dispExample}
\end{docCommand}


\begin{docCommand}{csvautobooklongtable}{\oarg{options}\marg{file name}}
  |csvautobooklongtable| is an abbreviation for the application of the option key
  \refKey{/csv/autobooklongtable} together with other \meta{options} to \refCom{csvloop}.
  This macro reads the whole CSV file denoted by \meta{file name}
  with an automated formatting.
  For application, the packages |booktabs| and |longtable| are required which have to be
  loaded in the preamble.
\begin{dispListing}
\csvautobooklongtable{grade.csv}
\end{dispListing}
\csvautobooklongtable{grade.csv}
\end{docCommand}



\clearpage

\begin{docCommand}{csvset}{\marg{options}}
  Sets \meta{options} for every following
  \refCom{csvreader} and \refCom{csvloop}. For example, this command may
  be used for style definitions.
\begin{dispExample}
\csvset{grade list/.style=
    {column names={name=\name,givenname=\firstname,grade=\grade}},
  passed/.style={filter not strcmp={\grade}{5.0}} }

The following students passed the test in mathematics:
\csvreader[grade list,passed]{grade.csv}{}{\firstname\ \name\ (\grade); }%
\end{dispExample}
\end{docCommand}


\begin{docCommand}{csvstyle}{\marg{key}\marg{options}}
  Abbreviation for |\csvset{|\meta{key}|/.style=|\marg{options}|}|
  to define a new style.
\end{docCommand}

\begin{docCommand}{csvnames}{\marg{key}\marg{assignments}}
  Abbreviation for |\csvset{|\meta{key}|/.style={column names=|\marg{assignments}|}}|
  to define additional \meta{assignments} of macros to columns.
\begin{dispExample}
\csvnames{grade list}{name=\name,givenname=\firstname,grade=\grade}
\csvstyle{passed}{filter not strcmp={\grade}{5.0}}

The following students passed the test in mathematics:
\csvreader[grade list,passed]{grade.csv}{}{\firstname\ \name\ (\grade); }%
\end{dispExample}
\end{docCommand}


\begin{docCommand}{csvheadset}{\marg{assignments}}
  For some special cases, this command can be used to change the
  \meta{assignments} of macros to columns during execution of
  \refCom{csvreader} and \refCom{csvloop}.
\begin{dispExample}
\csvreader{grade.csv}{}%
  { \csvheadset{name=\n} \fbox{\n}
    \csvheadset{givenname=\n} \ldots\ \fbox{\n}  }%
\end{dispExample}
\end{docCommand}

\clearpage

\begin{docCommand}{csviffirstrow}{\marg{then macros}\marg{else macros}}
  Inside the command list of \refCom{csvreader}, the \meta{then macros}
  are executed for the first data line, and the \meta{else macros}
  are executed for all following lines.
\begin{dispExample}
\csvreader[tabbing, head to column names, table head=\hspace*{3cm}\=\kill]%
  {grade.csv}{}%
  {\givenname~\name \> (\csviffirstrow{first entry!!}{following entry})}
\end{dispExample}
\end{docCommand}


\begin{docCommand}{csvifoddrow}{\marg{then macros}\marg{else macros}}
  Inside the command list of \refCom{csvreader}, the \meta{then macros}
  are executed for odd-numbered   data lines, and the \meta{else macros}
  are executed for even-numbered lines.
\begin{dispExample}
\csvreader[head to column names,tabular=|l|l|l|l|,
  table head=\hline\bfseries \# & \bfseries Name & \bfseries Grade\\\hline,
  table foot=\hline]{grade.csv}{}{%
    \csvifoddrow{\slshape\thecsvrow & \slshape\name, \givenname & \slshape\grade}%
    {\bfseries\thecsvrow & \bfseries\name, \givenname & \bfseries\grade}}
\end{dispExample}

The |\csvifoddrow| macro may be used for striped tables:

\begin{dispExample}
% This example needs the xcolor package
\csvreader[head to column names,tabular=rlcc,
  table head=\hline\rowcolor{red!50!black}\color{white}\# & \color{white}Person
    & \color{white}Matr.~No. & \color{white}Grade,
  late after head=\\\hline\rowcolor{yellow!50},
  late after line=\csvifoddrow{\\\rowcolor{yellow!50}}{\\\rowcolor{red!25}}]%
  {grade.csv}{}%
  {\thecsvrow & \givenname~\name & \matriculation & \grade}%
\end{dispExample}

\enlargethispage*{1cm}
Alternatively, |\rowcolors| from the |xcolor| package can be used for this
purpose:

\begin{dispExample}
% This example needs the xcolor package
\csvreader[tabular=rlcc, before table=\rowcolors{2}{red!25}{yellow!50},
  table head=\hline\rowcolor{red!50!black}\color{white}\# & \color{white}Person
    & \color{white}Matr.~No. & \color{white}Grade\\\hline,
  head to column names]{grade.csv}{}%
  {\thecsvrow & \givenname~\name & \matriculation & \grade}%
\end{dispExample}
\end{docCommand}

\clearpage

\begin{docCommand}{csvfilteraccept}{}
  All following consistent data lines will be accepted and processed.
  This command overwrites all previous filter settings and may be used
  inside \refKey{/csv/full filter} to implement
  an own filtering rule together with |\csvfilterreject|.
\begin{dispExample}
\csvreader[autotabular,
  full filter=\ifcsvstrcmp{\csvcoliv}{m}{\csvfilteraccept}{\csvfilterreject}
  ]{grade.csv}{}{\csvlinetotablerow}%
\end{dispExample}
\end{docCommand}


\begin{docCommand}{csvfilterreject}{}
  All following data lines will be ignored.
  This command overwrites all previous filter settings.
\end{docCommand}


\begin{docCommand}{csvline}{}
  This macro contains the current and unprocessed data line.
\begin{dispExample}
\csvreader[no head, tabbing, table head=\textit{line XX:}\=\kill]%
  {grade.csv}{}{\textit{line \thecsvrow:} \> \csvline}%
\end{dispExample}
\end{docCommand}


\begin{docCommand}{thecsvrow}{}
  Typesets the current data line number. This is the
  current number of accepted data lines without the header line.
  The \LaTeX\ counter |csvrow| can be addressed directly in the usual way,
  e.\,g. by |\roman{csvrow}|.
\end{docCommand}


\begin{docCommand}{thecsvinputline}{}
  Typesets the current file line number. This is the
  current number of all data lines including the header line.
  The \LaTeX\ counter |csvinputline| can be addressed directly in the usual way,
  e.\,g. by |\roman{csvinputline}|.
\begin{dispExample}
\csvreader[no head, filter test=\ifnumequal{\thecsvinputline}{3}]%
  {grade.csv}{}%
  {The line with number \thecsvinputline\ contains: \csvline}%
\end{dispExample}
\end{docCommand}


\begin{docCommand}[doc updated=2016-07-01]{csvlinetotablerow}{}
  Typesets the current processed data line with |&| between the entries.
  %Most users will never apply this command.
\end{docCommand}



\clearpage
\section{Option Keys}\label{sec:schluessel}%
For the \meta{options} in \refCom{csvreader} respectively \refCom{csvloop}
the following |pgf| keys can be applied. The key tree path |/csv/| is not
to be used inside these macros.


\subsection{Command Definition}%--------%[[

\begin{docCsvKey}{before reading}{=\meta{code}}{no default, initially empty}
  Sets the \meta{code} to be executed before the CSV file is processed.
\end{docCsvKey}

\begin{docCsvKey}{after head}{=\meta{code}}{no default, initially empty}
  Sets the \meta{code} to be executed after the header line is read.
\end{docCsvKey}

\begin{docCsvKey}{before filter}{=\meta{code}}{no default, initially empty}
  Sets the \meta{code} to be executed after reading and consistency checking
  of a data line. They are executed before any filter condition is checked,
  see \refKey{/csv/filter}.
  Also see \refKey{/csv/full filter}.
\end{docCsvKey}

\begin{docCsvKey}{late after head}{=\meta{code}}{no default, initially empty}
  Sets the \meta{code} to be executed after reading and disassembling
  of the first accepted data line. They are executed before further processing
  of this line.
\end{docCsvKey}

\begin{docCsvKey}{late after line}{=\meta{code}}{no default, initially empty}
  Sets the \meta{code} to be executed after reading and disassembling
  of the next accepted data line (after \refKey{/csv/before filter}).
  They are executed before further processing of this next line.
  |late after line| overwrites |late after first line| and |late after last line|.
  Note that table options like \refKey{/csv/tabular} set this key to |\\|
  automatically.
\end{docCsvKey}


\begin{docCsvKey}{late after first line}{=\meta{code}}{no default, initially empty}
  Sets the \meta{code} to be executed after reading and disassembling
  of the second accepted data line instead of \refKey{/csv/late after line}.
  This key has to be set after |late after line|.
\end{docCsvKey}


\begin{docCsvKey}{late after last line}{=\meta{code}}{no default, initially empty}
  Sets the \meta{code} to be executed after processing of the last
  accepted data line instead of \refKey{/csv/late after line}.
  This key has to be set after |late after line|.
\end{docCsvKey}


\begin{docCsvKey}{before line}{=\meta{code}}{no default, initially empty}
  Sets the \meta{code} to be executed after \refKey{/csv/late after line}
  and before \refKey{/csv/command}.
  |before line| overwrites |before first line|.
\end{docCsvKey}


\begin{docCsvKey}{before first line}{=\meta{code}}{no default, initially empty}
  Sets the \meta{code} to be executed instead of \refKey{/csv/before line}
  for the first accepted data line.
  This key has to be set after |before line|.
\end{docCsvKey}


\begin{docCsvKey}{command}{=\meta{code}}{no default, initially \cs{csvline}}
  Sets the \meta{code} to be executed for every accepted data line.
  They are executed between \refKey{/csv/before line} and \refKey{/csv/after line}.
\end{docCsvKey}


\begin{docCsvKey}{after line}{=\meta{code}}{no default, initially empty}
  Sets the \meta{code} to be executed for every accepted data line
  after \refKey{/csv/command}.
  |after line| overwrites |after first line|.
\end{docCsvKey}


\begin{docCsvKey}{after first line}{=\meta{code}}{no default, initially empty}
  Sets the \meta{code} to be executed instead of \refKey{/csv/after line}
  for the first accepted data line.
  This key has to be set after |after line|.
\end{docCsvKey}

\begin{docCsvKey}{after reading}{=\meta{code}}{no default, initially empty}
  Sets the \meta{code} to be executed after the CSV file is processed.
\end{docCsvKey}


\begin{dispExample}
\csvreader[
  before reading        = \meta{before reading}\\,
  after head            = \meta{after head},
  before filter         = \\\meta{before filter},
  late after head       = \meta{late after head},
  late after line       = \meta{late after line},
  late after first line = \meta{late after first line},
  late after last line  = \\\meta{late after last line},
  before line           = \meta{before line},
  before first line     = \meta{before first line},
  after line            = \meta{after line},
  after first line      = \meta{after first line},
  after reading         = \\\meta{after reading}
    ]{grade.csv}{name=\name}{\textbf{\name}}%
\end{dispExample}

Additional command definition keys are provided for the supported tables,
see Section~\ref{subsec:tabsupport} from page~\pageref{subsec:tabsupport}.

\clearpage
\subsection{Header Processing and Column Name Assignment}%

\begin{docCsvKey}{head}{\colOpt{=true\textbar false}}{default |true|, initially |true|}
  If this key is set, the first line of the CSV file is treated as a header
  line which can be used for column name assignments.
\end{docCsvKey}


\begin{docCsvKey}{no head}{}{no value}
  Abbreviation for |head=false|, i.\,e. the first line of the CSV file is
  treated as data line.
  Note that this option cannot be used in combination with
  \refCom{csvautotabular}, \refKey{/csv/autotabular}, and similar automated commands/options.
  See Section~\ref{noheader} on page~\pageref{noheader} for assistance.
\end{docCsvKey}

\begin{docCsvKey}{column names}{=\meta{assignments}}{no default, initially empty}
  Adds some new \meta{assignments} of macros to columns in key value syntax.
  Existing assignments are kept.
\end{docCsvKey}


\begin{docCsvKey}{column names reset}{}{no value}
  Clears all assignments of macros to columns.
\end{docCsvKey}


\begin{docCsvKey}{head to column names}{\colOpt{=true\textbar false}}{default |true|, initially |false|}
  If this key is set, the entries of the header line are used automatically
  as macro names for the columns. This option can be used only, if
  the header entries do not contain spaces and special characters to be
  used as feasible \LaTeX\ macro names.
  Note that the macro definition is \emph{global} and may therefore override
  existing macros for the rest of the document. Adding
  \refKey{/csv/head to column names prefix} may help to avoid unwanted
  overrides.
\end{docCsvKey}


\begin{docCsvKey}[][doc new=2019-07-16]{head to column names prefix}{=\meta{text}}{no default, initially empty}
  The given \meta{text} is prefixed to the name of all macros generated by
  \refKey{/csv/head to column names}. For example, if you use the settings
\begin{dispListing}
    head to column names,
    head to column names prefix=MY,
\end{dispListing}
  a header entry |section| will generate the corresponding macro
  |\MYsection| instead of destroying the standard \LaTeX\ |\section| macro.
\end{docCsvKey}


\clearpage
\subsection{Consistency Check}\label{sec:consistency}%

\begin{docCsvKey}{check column count}{\colOpt{=true\textbar false}}{default |true|, initially |true|}
  This key defines, wether the number of entries in a data line is checked against
  an expected value or not.\\
  If |true|, every non consistent line is ignored without announcement.\\
  If |false|, every line is accepted and may produce an error during
  further processing.
\end{docCsvKey}


\begin{docCsvKey}{no check column count}{}{no value}
  Abbreviation for |check column count=false|.
\end{docCsvKey}


\begin{docCsvKey}{column count}{=\meta{number}}{no default}
  Sets the \meta{number} of feasible entries per data line.
  This setting is only useful in connection with \refKey{/csv/no head},
  since \meta{number} would be replaced by the number of entries in the
  header line otherwise.
\end{docCsvKey}


\begin{docCsvKey}{on column count error}{=\meta{code}}{no default, initially empty}
  \meta{code} to be executed for unfeasible data lines.
\end{docCsvKey}


\begin{docCsvKey}{warn on column count error}{}{style, no value}
  Display of a warning for unfeasible data lines.
\end{docCsvKey}


\clearpage
\subsection{Filtering}%

\begin{docCsvKey}[][doc new=2016-07-01]{filter test}{=\meta{condition}}{no default}
  Only data lines which fulfill a logical \meta{condition} are accepted.
  For the \meta{condition}, every single test normally employed like
\begin{dispListing}
\iftest{some testing}{true}{false}
\end{dispListing}
  can be used as
\begin{dispListing}
filter test=\iftest{some testing},
\end{dispListing}
  For |\iftest|, tests from the |etoolbox| package like
  |\ifnumcomp|, |\ifdimgreater|, etc. and from \Fullref{sec:stringtests} can be used.

\begin{dispExample}
\csvreader[head to column names,tabular=llll,
  table head=\toprule & \bfseries Name & \bfseries Matr & \bfseries Grade\\\midrule,
  table foot=\bottomrule,
  %>> list only matriculation numbers greater than 20000 <<
  filter test=\ifnumgreater{\matriculation}{20000},
  ]{grade.csv}{}{%
    \thecsvrow & \slshape\name, \givenname & \matriculation & \grade}
\end{dispExample}
\end{docCsvKey}


\begin{docCsvKey}{filter strcmp}{=\marg{stringA}\marg{stringB}}{style, no default}
  Only lines where \meta{stringA} and \meta{stringB} are equal after expansion
  are accepted.
  The implementation is done with \refCom{ifcsvstrcmp}.
\end{docCsvKey}


\begin{docCsvKey}{filter not strcmp}{=\marg{stringA}\marg{stringB}}{style, no default}
  Only lines where \meta{stringA} and \meta{stringB} are not equal after expansion
  are accepted.
  The implementation is done with \refCom{ifcsvnotstrcmp}.
\end{docCsvKey}


\begin{docCsvKey}[][doc new=2016-07-01]{filter expr}{=\meta{condition}}{no default}
  Only data lines which fulfill a logical \meta{condition} are accepted.
  For the \meta{condition}, every boolean expression
  from the |etoolbox| package is feasible.
  To preprocess the data line before testing the \meta{condition},
  the option key \refKey{/csv/before filter} can be used.
\begin{dispExample}
\csvreader[head to column names,tabular=llll,
  table head=\toprule & \bfseries Name & \bfseries Matr & \bfseries Grade\\\midrule,
  table foot=\bottomrule,
  %>> list only matriculation numbers greater than 20000
  %   and grade less than 4.0 <<
  filter expr={     test{\ifnumgreater{\matriculation}{20000}}
                and test{\ifdimless{\grade pt}{4.0pt}}          },
  ]{grade.csv}{}{%
    \thecsvrow & \slshape\name, \givenname & \matriculation & \grade}
\end{dispExample}
\end{docCsvKey}

\clearpage
\begin{docCsvKey}[][doc new=2016-07-01]{filter ifthen}{=\meta{condition}}{no default}
  Only data lines which fulfill a logical \meta{condition} are accepted.
  For the \meta{condition}, every term from the |ifthen| package
  is feasible.
  To preprocess the data line before testing the \meta{condition},
  the option key \refKey{/csv/before filter} can be used.

\begin{dispExample}
\csvreader[head to column names,tabular=llll,
  table head=\toprule & \bfseries Name & \bfseries Matr & \bfseries Grade\\\midrule,
  table foot=\bottomrule,
  %>> list only female persons <<
  filter ifthen=\equal{\gender}{f},
  ]{grade.csv}{}{%
    \thecsvrow & \slshape\name, \givenname & \matriculation & \grade}
\end{dispExample}

\end{docCsvKey}


\begin{docCsvKey}{filter}{=\meta{condition}}{no default}
  Alias for \refKey{/csv/filter ifthen}.
\end{docCsvKey}

\begin{docCsvKey}{filter equal}{=\marg{stringA}\marg{stringB}}{style, no default}
  Only lines where \meta{stringA} and \meta{stringB} are equal after expansion
  are accepted.
  The implementation is done with the |ifthen| package.
\end{docCsvKey}


\begin{docCsvKey}{filter not equal}{=\marg{stringA}\marg{stringB}}{style, no default}
  Only lines where \meta{stringA} and \meta{stringB} are not equal after expansion
  are accepted.
  The implementation is done with the |ifthen| package.
\end{docCsvKey}



\begin{docCsvKey}{no filter}{}{no value, initially set}
  Clears a set filter.
\end{docCsvKey}


\begin{docCsvKey}{filter accept all}{}{no value, initially set}
  Alias for |no filter|. All consistent data lines are accepted.
\end{docCsvKey}


\begin{docCsvKey}{filter reject all}{}{no value}
  All data line are ignored.
\end{docCsvKey}



\enlargethispage*{2cm}
\begin{docCsvKey}[][doc new=2016-07-01]{full filter}{=\meta{code}}{no default}
  Technically, this key is an alias for \refKey{/csv/before filter}.
  Philosophically, \refKey{/csv/before filter} computes something before
  a filter condition is set, but \refKey{/csv/full filter} should implement
  the full filtering. Especially, \refCom{csvfilteraccept} or
  \refCom{csvfilterreject} \emph{should} be set inside the \meta{code}.
\begin{dispExample}
\csvreader[head to column names,tabular=llll,
  table head=\toprule & \bfseries Name & \bfseries Matr & \bfseries Grade\\\midrule,
  table foot=\bottomrule,
  %>> list only matriculation numbers greater than 20000
  %   and grade less than 4.0 <<
  full filter=\ifnumgreater{\matriculation}{20000}
                {\ifdimless{\grade pt}{4.0pt}{\csvfilteraccept}{\csvfilterreject}}
                {\csvfilterreject},
  ]{grade.csv}{}{%
    \thecsvrow & \slshape\name, \givenname & \matriculation & \grade}
\end{dispExample}
\end{docCsvKey}



%]]


\clearpage
\subsection{Table Support}\label{subsec:tabsupport}%--------%[[

\begin{docCsvKey}{tabular}{=\meta{table format}}{style, no default}
  Surrounds the CSV processing with |\begin{tabular}|\marg{table format}
  at begin and with |\end{tabular}| at end.
Additionally, the commands defined by the key values of
  \refKey{/csv/before table}, \refKey{/csv/table head}, \refKey{/csv/table foot},
  and \refKey{/csv/after table} are executed at the appropriate places.
\end{docCsvKey}


\begin{docCsvKey}{centered tabular}{=\meta{table format}}{style, no default}
  Like \refKey{/csv/tabular} but inside an additional |center| environment.
\end{docCsvKey}


\begin{docCsvKey}{longtable}{=\meta{table format}}{style, no default}
  Like \refKey{/csv/tabular} but for the |longtable| environment.
  This requires the package |longtable| (not loaded automatically).
\end{docCsvKey}


\begin{docCsvKey}{tabbing}{}{style, no value}
  Like \refKey{/csv/tabular} but for the |tabbing| environment.
\end{docCsvKey}


\begin{docCsvKey}{centered tabbing}{}{style, no value}
  Like \refKey{/csv/tabbing} but inside an additional |center| environment.
\end{docCsvKey}


\begin{docCsvKey}{no table}{}{style, no value}
  Deactivates |tabular|, |longtable|, and |tabbing|.
\end{docCsvKey}


\begin{docCsvKey}{before table}{=\meta{code}}{no default, initially empty}
  Sets the \meta{code} to be executed before |\begin{tabular}| or before |\begin{longtable}|
  or before |\begin{tabbing}|, respectively.
\end{docCsvKey}


\begin{docCsvKey}{table head}{=\meta{code}}{no default, initially empty}
  Sets the \meta{code} to be executed after |\begin{tabular}| or after |\begin{longtable}|
  or after |\begin{tabbing}|, respectively.
\end{docCsvKey}


\begin{docCsvKey}{table foot}{=\meta{code}}{no default, initially empty}
  Sets the \meta{code} to be executed before |\end{tabular}| or before |\end{longtable}|
  or before |\end{tabbing}|, respectively.
\end{docCsvKey}


\begin{docCsvKey}{after table}{=\meta{code}}{no default, initially empty}
  Sets the \meta{code} to be executed after |\end{tabular}| or after |\end{longtable}|
  or after |\end{tabbing}|, respectively.
\end{docCsvKey}

\bigskip

The following |auto| options are the counterparts for the respective quick
overview commands like \refCom{csvautotabular}. They are listed for
completeness, but are unlikely to be used directly.

\begin{docCsvKey}{autotabular}{=\meta{file name}}{no default}
  Reads the whole CSV file denoted \meta{file name} with an automated formatting.
\end{docCsvKey}


\begin{docCsvKey}{autolongtable}{=\meta{file name}}{no default}
  Reads the whole CSV file denoted \meta{file name} with an automated formatting
  using the required |longtable| package.
\end{docCsvKey}


\begin{docCsvKey}{autobooktabular}{=\meta{file name}}{no default}
  Reads the whole CSV file denoted \meta{file name} with an automated formatting
  using the required |booktabs| package.
\end{docCsvKey}


\begin{docCsvKey}{autobooklongtable}{=\meta{file name}}{no default}
  Reads the whole CSV file denoted \meta{file name} with an automated formatting
  using the required |booktabs| and |longtable| packages.
\end{docCsvKey}


\clearpage
\subsection{Special Characters}\label{subsec:specchar}
Be default, the CSV content is treated like normal \LaTeX\ text, see
Subsection~\ref{macrocodexample} on page~\pageref{macrocodexample}.
But, \TeX\ special characters of the CSV content may also be interpreted
as normal characters, if one or more of the following options are used.

\begin{docCsvKey}{respect tab}{\colOpt{=true\textbar false}}{default |true|, initially |false|}
  If this key is set, every
  tabulator sign
  inside the CSV content is a normal character.
\end{docCsvKey}

\begin{docCsvKey}{respect percent}{\colOpt{=true\textbar false}}{default |true|, initially |false|}
  If this key is set, every
  percent sign \verbbox{\%}
  inside the CSV content is a normal character.
\end{docCsvKey}

\begin{docCsvKey}{respect sharp}{\colOpt{=true\textbar false}}{default |true|, initially |false|}
  If this key is set, every
  sharp sign \verbbox{\#}
  inside the CSV content is a normal character.
\end{docCsvKey}

\begin{docCsvKey}{respect dollar}{\colOpt{=true\textbar false}}{default |true|, initially |false|}
  If this key is set, every
  dollar sign \verbbox{\$}
  inside the CSV content is a normal character.
\end{docCsvKey}

\begin{docCsvKey}{respect and}{\colOpt{=true\textbar false}}{default |true|, initially |false|}
  If this key is set, every
  and sign \verbbox{\&}
  inside the CSV content is a normal character.
\end{docCsvKey}

\begin{docCsvKey}{respect backslash}{\colOpt{=true\textbar false}}{default |true|, initially |false|}
  If this key is set, every
  backslash sign \verbbox{\textbackslash}
  inside the CSV content is a normal character.
\end{docCsvKey}

\begin{docCsvKey}{respect underscore}{\colOpt{=true\textbar false}}{default |true|, initially |false|}
  If this key is set, every
  underscore sign \verbbox{\_}
  inside the CSV content is a normal character.
\end{docCsvKey}

\begin{docCsvKey}{respect tilde}{\colOpt{=true\textbar false}}{default |true|, initially |false|}
  If this key is set, every
  tilde sign \verbbox{\textasciitilde}
  inside the CSV content is a normal character.
\end{docCsvKey}

\begin{docCsvKey}{respect circumflex}{\colOpt{=true\textbar false}}{default |true|, initially |false|}
  If this key is set, every
  circumflex sign \verbbox{\textasciicircum}
  inside the CSV content is a normal character.
\end{docCsvKey}

\begin{docCsvKey}{respect leftbrace}{\colOpt{=true\textbar false}}{default |true|, initially |false|}
  If this key is set, every
  left brace sign \verbbox{\textbraceleft}
  inside the CSV content is a normal character.
\end{docCsvKey}

\begin{docCsvKey}{respect rightbrace}{\colOpt{=true\textbar false}}{default |true|, initially |false|}
  If this key is set, every
  right brace sign \verbbox{\textbraceright}
  inside the CSV content is a normal character.
\end{docCsvKey}

\begin{docCsvKey}{respect all}{}{style, no value, initially unset}
  Set all special characters from above to normal characters. This means
  a quite verbatim interpretation of the CSV content.
\end{docCsvKey}

\begin{docCsvKey}{respect none}{}{style, no value, initially set}
  Do not change any special character from above to normal character.
\end{docCsvKey}

\clearpage
\subsection{Separators}\label{sec:separators}%
\begin{docCsvKey}{separator}{=\meta{sign}}{no default, initially |comma|}
  \catcode `|=12
  Sets the \meta{sign} which is treates as separator between the data values
  of a data line. Feasible values are:
  \begin{itemize}
  \item\docValue{comma}: This is the initial value with '\texttt{,}' as separator.
  \medskip

  \item\docValue{semicolon}: Sets the separator to '\texttt{;}'.
\begin{dispExample}
% \usepackage{tcolorbox} for tcbverbatimwrite
\begin{tcbverbatimwrite}{testsemi.csv}
  name;givenname;matriculation;gender;grade
  Maier;Hans;12345;m;1.0
  Huber;Anna;23456;f;2.3
  Weißbäck;Werner;34567;m;5.0
\end{tcbverbatimwrite}

\csvautobooktabular[separator=semicolon]{testsemi.csv}
\end{dispExample}
\medskip

\item\docValue{pipe}: Sets the separator to '\texttt{|}'.
\begin{dispExample}
% \usepackage{tcolorbox} for tcbverbatimwrite
\begin{tcbverbatimwrite}{pipe.csv}
  name|givenname|matriculation|gender|grade
  Maier|Hans|12345|m|1.0
  Huber|Anna|23456|f|2.3
  Weißbäck|Werner|34567|m|5.0
\end{tcbverbatimwrite}

\csvautobooktabular[separator=pipe]{pipe.csv}
\end{dispExample}
\medskip

\item\docValue{tab}: Sets the separator to the tabulator sign.
  Automatically, \refKey{/csv/respect tab} is set also.
  \end{itemize}
\end{docCsvKey}

\clearpage
\subsection{Miscellaneous}%

\begin{docCsvKey}{every csv}{}{style, initially empty}
  A style definition which is used for every following CSV file.
  This definition can be overwritten with user code.
\begin{dispListing}
% Sets a warning message for unfeasible data lines.
\csvset{every csv/.style={warn on column count error}}
% Alternatively:
\csvstyle{every csv}{warn on column count error}
\end{dispListing}
\end{docCsvKey}

\begin{docCsvKey}{default}{}{style}
  A style definition which is used for every following CSV file which
  resets all settings to default values\footnote{\texttt{default} is used
  because of the global nature of most settings.}.
  This key should not be used or changed by the user if there is not a
  really good reason (and you know what you do).
\end{docCsvKey}


\begin{docCsvKey}{file}{=\meta{file name}}{no default, initially |unknown.csv|}
  Sets the \meta{file name} of the CSV file to be processed.
\end{docCsvKey}


\begin{docCsvKey}{preprocessed file}{=\meta{file name}}{no default, initially \texttt{\textbackslash\detokenize{jobname_sorted.csv}}}
  Sets the \meta{file name} of the CSV file which is the output of a
  preprocessor.
\end{docCsvKey}


\begin{docCsvKey}{preprocessor}{=\meta{macro}}{no default}
  Defines a preprocessor for the given CSV file.
  The \meta{macro} has to have two mandatory arguments. The first argument
  is the original CSV file which is set by \refKey{/csv/file}.
  The second argument is the preprocessed CSV file
  which is set by \refKey{/csv/preprocessed file}.\par\smallskip
  Typically, the \meta{macro} may call an external program which preprocesses
  the original CSV file (e.\,g. sorting the file) and creates the
  preprocessed CSV file. The later file is used by \refCom{csvreader}
  or \refCom{csvloop}.
\begin{dispListing}
\newcommand{\mySortTool}[2]{%
  % call to an external program to sort file #1 with resulting file #2
}

\csvreader[%
  preprocessed file=\jobname_sorted.csv,
  preprocessor=\mySortTool,
  ]{some.csv}{}{%
  % do something
}
\end{dispListing}
See Subsection~\ref{sec:Sorting} on page~\pageref{sec:Sorting} for a
concrete sorting preprocessing implemented with an external tool.
\end{docCsvKey}


\begin{docCsvKey}{no preprocessing}{}{style, no value, initially set}
  Clears any preprocessing, i.\,e. preprocessing is switched of.
\end{docCsvKey}



\clearpage
\subsection{Sorting}\label{sec:Sorting}%
\TeX/\LaTeX\ was not born under a sorting planet. |csvsimple-legacy| provides no
sorting of data lines by \LaTeX-methods since sorting can be done much faster
and much better by external tools.

First, one should consider the appropriate \emph{place} for sorting:
\begin{itemize}
\item CSV files may be sorted by a tool \emph{before} the \LaTeX\ document is processed
  at all. If the CSV data is not likely to change, this is the most efficient method.
\item CSV files may be sorted by a tool every time before the \LaTeX\ document is compiled.
  This could be automated by a shell script or some processing tool like |arara|.
\item CSV files may be sorted on-the-fly by a tool during compilation of
  a \LaTeX\ document. This is the most elegant but not the most efficient way.
\end{itemize}

The first two methods are decoupled from anything concerning |csvsimple-legacy|.
For the third method, the \refKey{/csv/preprocessor} option is made for.
This allows to access an external tool for sorting.
\emph{Which tool} is your choice.

\csvsorter\ was written as a companion tool for |csvsimple|.
It is an open source Java command-line tool for sorting CSV files, available at\\
\url{http://T-F-S.github.io/csvsorter/}\quad or\quad
\url{https://github.com/T-F-S/csvsorter}

It can be
used for all three sorting approaches described above.
There is special support for on-the-fly sorting with \csvsorter\ using the
following options.

\begin{enumerate}\bfseries
\item To use the sorting options, you have to install \csvsorter\ before!\\
  |csvsimple| v1.12 or newer needs \csvsorter\ v0.94 of newer!
\item You have to give permission to call external tools during
  compilation, i.\,e.\ the command-line options for |latex| have to include
  |-shell-escape|.
\end{enumerate}

\bigskip

\begin{docCsvKey}{csvsorter command}{=\meta{system command}}{no default, initially |csvsorter|}
  The \meta{system command} specifies the system call for \csvsorter\ (without the options).
  If \csvsorter\ was completely installed following its documentation, there is
  nothing to change here. If the |csvsorter.jar| file is inside the same
  directory as the \LaTeX\ source file, you may configure:% preferrably inside the preamble:
\begin{dispListing}
\csvset{csvsorter command=java -jar csvsorter.jar}
\end{dispListing}
\end{docCsvKey}

\begin{docCsvKey}{csvsorter configpath}{=\meta{path}}{no default, initially |.|}
  Sorting with \csvsorter\ is done using XML configuration files. If these files
  are not stored inside the same directory as the \LaTeX\ source file, a
  \meta{path} to access them can be configured:
\begin{dispListing}
\csvset{csvsorter configpath=xmlfiles}
\end{dispListing}
  Here, the configuration files would be stored in a subdirectory named |xmlfiles|.
\end{docCsvKey}

\begin{docCsvKey}{csvsorter log}{=\meta{file name}}{no default, initially |csvsorter.log|}
  Sets the log file of \csvsorter\ to the given \meta{file name}.
\begin{dispListing}
\csvset{csvsorter log=outdir/csvsorter.log}
\end{dispListing}
  Here, the log file is written to a subdirectory named |outdir|.
\end{docCsvKey}

\clearpage
\begin{docCsvKey}{csvsorter token}{=\meta{file name}}{no default, initially |\textbackslash jobname.csvtoken|}
  Sets \meta{file name} as token file. This is an auxiliary file which
  communicates the success of \csvsorter\ to |csvsimple|.
\begin{dispListing}
\csvset{csvsorter log=outdir/\jobname.csvtoken}
\end{dispListing}
  Here, the token file is written to a subdirectory named |outdir|.
\end{docCsvKey}


\begin{docCsvKey}{sort by}{=\meta{file name}}{style, initially unset}
  The \meta{file name} denotes an XML configuration file for \csvsorter.
  Setting this option inside \refCom{csvreader} or
  \refCom{csvloop} will issue a system call to \csvsorter.
  \begin{itemize}
  \item \csvsorter\ uses the given CSV file as input file.
  \item \csvsorter\ uses \meta{file name} as configuration file.
  \item The output CSV file is denoted by \refKey{/csv/preprocessed file}
     which is by default \texttt{\textbackslash\detokenize{jobname_sorted.csv}}.
     This output file is this actual file processed by \refCom{csvreader} or \refCom{csvloop}.
  \item \csvsorter\ also generates a log file denoted by \refKey{/csv/csvsorter log} which is by default |csvsorter.log|.
  \end{itemize}

\par\medskip\textbf{First example:}
  To sort our example |grade.csv| file according to |name| and |givenname|, we
  use the following XML configuration file. Since \csvsorter\ uses double quotes
  as default brackets for column values, we remove bracket recognition to avoid
  a clash with the escaped umlauts of the example CSV file.\par\smallskip

\xmllisting{namesort}
\begin{dispExample}
% \usepackage{booktabs}
\csvreader[sort by=namesort.xml,
  head to column names,
  tabular=>{\color{red}}lllll,
  table head=\toprule Name & Given Name & Matriculation & Gender & Grade\\\midrule,
  table foot=\bottomrule]
  {grade.csv}{}{\csvlinetotablerow}
\end{dispExample}

\clearpage\textbf{Second example:}
  To sort our example |grade.csv| file according to |grade|, we
  use the following XML configuration file. Further, persons with the same |grade|
  are sorted by |name| and |givenname|. Since \csvsorter\ uses double quotes
  as default brackets for column values, we remove bracket recognition to avoid
  a clash with the escaped umlauts of the example CSV file.\par\smallskip

\xmllisting{gradesort}
\begin{dispExample}
% \usepackage{booktabs}
\csvreader[sort by=gradesort.xml,
  head to column names,
  tabular=llll>{\color{red}}l,
  table head=\toprule Name & Given Name & Matriculation & Gender & Grade\\\midrule,
  table foot=\bottomrule]
  {grade.csv}{}{\csvlinetotablerow}
\end{dispExample}

\clearpage\textbf{Third example:}
  To generate a matriculation/grade list, we sort our example |grade.csv| file
  using the following XML configuration file.
  Again, since \csvsorter\ uses double quotes
  as default brackets for column values, we remove bracket recognition to avoid
  a clash with the escaped umlauts of the example CSV file.\par\smallskip

\xmllisting{matriculationsort}
\begin{dispExample}
% \usepackage{booktabs}
\csvreader[sort by=matriculationsort.xml,
  head to column names,
  tabular=>{\color{red}}ll,
  table head=\toprule Matriculation & Grade\\\midrule,
  table foot=\bottomrule]
  {grade.csv}{}{\matriculation & \grade}
\end{dispExample}
\end{docCsvKey}


\clearpage
\begin{docCsvKey}{new sorting rule}{=\marg{name}\marg{file name}}{style, initially unset}
This is a convenience option to generate a new shortcut for often used
\refKey{/csv/sort by} applications. It also adds a more semantic touch.
The new shortcut option is
\tcbox[on line,size=small,colback=white,colframe=red]{|sort by| \meta{name}} which expands to
\tcbox[on line,size=small,colback=white,colframe=red]{|sort by=|\marg{file name}}.\par\medskip

Consider the following example:
\begin{dispExample}
\csvautotabular[sort by=namesort.xml]{grade.csv}
\end{dispExample}
A good place for setting up a new sorting rule would be inside the preamble:

\csvset{new sorting rule={name}{namesort.xml}}
\begin{dispListing}
\csvset{new sorting rule={name}{namesort.xml}}
\end{dispListing}

Now, we can use the new rule:
\begin{dispExample}
\csvautotabular[sort by name]{grade.csv}
\end{dispExample}

\end{docCsvKey}


\clearpage
\section{String Tests}\label{sec:stringtests}%

The following string tests are complementing the string tests
from the |etoolbox| package. They all do the same, i.e.,
comparing expanded strings for equality.
\begin{itemize}
\item\refCom{ifcsvstrcmp} is the most efficient method, because it uses
  native compiler string comparison (if available).
\item\refCom{ifcsvstrequal} does not rely on a compiler. It also is the
  fallback implementation for \refCom{ifcsvstrcmp}, if there is no
  native comparison method.
\item\refCom{ifcsvprostrequal} is possibly more failsafe than the other two
  string tests. It may be used, if strings contain dirty things like |\textbf{A}|.
\end{itemize}
\medskip

\begin{docCommand}[doc new=2016-07-01]{ifcsvstrcmp}{\marg{stringA}\marg{stringB}\marg{true}\marg{false}}
  Compares two strings and executes \meta{true} if they are equal, and \meta{false} otherwise.
  The comparison is done using |\pdfstrcmp|, if compilation is done with pdf\LaTeX.
  The comparison is done using |\pdf@strcmp|, if the package |pdftexcmds| is
  loaded and compilation is done with lua\LaTeX\ or Xe\LaTeX.
  Otherwise, \refCom{ifcsvstrcmp} is identical to \refCom{ifcsvstrequal}.
  This command cannot be used inside the preamble.
\end{docCommand}


\begin{docCommand}[doc new=2016-07-01]{ifcsvnotstrcmp}{\marg{stringA}\marg{stringB}\marg{true}\marg{false}}
  Compares two strings and executes \meta{true} if they are \emph{not} equal, and \meta{false} otherwise.
  The implementation uses \refCom{ifcsvstrcmp}.
\end{docCommand}


\begin{docCommand}[doc new=2016-07-01]{ifcsvstrequal}{\marg{stringA}\marg{stringB}\marg{true}\marg{false}}
  Compares two strings and executes \meta{true} if they are equal, and \meta{false} otherwise.
  The strings are expanded with |\edef| in the test.
\end{docCommand}

\begin{docCommand}[doc new=2016-07-01]{ifcsvprostrequal}{\marg{stringA}\marg{stringB}\marg{true}\marg{false}}
  Compares two strings and executes \meta{true} if they are equal, and \meta{false} otherwise.
  The strings are expanded with |\protected@edef| in the test, i.e. parts of the
  strings which are protected stay unexpanded.
\end{docCommand}



\clearpage
\section{Examples}%

\subsection{A Serial Letter}%
In this example, a serial letter is to be written to all persons with
addresses from the following CSV file. Deliberately, the file content is
not given in very pretty format.

%-- file embedded for simplicity --
\begin{tcbverbatimwrite}{address.csv}
name,givenname,gender,degree,street,zip,location,bonus
Maier,Hans,m,,Am Bachweg 17,10010,Hopfingen,20
    % next line with a comma in curly braces
Huber,Erna,f,Dr.,{Moosstraße 32, Hinterschlag},10020,Örtingstetten,30
Weißbäck,Werner,m,Prof. Dr.,Brauallee 10,10030,Klingenbach,40
    % this line is ignored %
  Siebener ,  Franz,m,   ,  Blaumeisenweg 12  , 10040 ,  Pardauz , 50
    % preceding and trailing spaces in entries are removed %
Schmitt,Anton,m,,{\AE{}lfred-Esplanade, T\ae{}g 37}, 10050,\OE{}resung,60
\end{tcbverbatimwrite}
%-- end embedded file --

\csvlisting{address}

Firstly, we survey the file content quickly using
|\csvautotabular|.
As can be seen, unfeasible lines are ignored automatically.

\begin{dispExample}
\tiny\csvautotabular{address.csv}
\end{dispExample}

Now, we create the serial letter where every feasible data line produces
an own page. Here, we simulate the page by a |tcolorbox| (from the package
|tcolorbox|).
For the gender specific salutations, an auxiliary macro |\ifmale| is
introduced.

\begin{dispExample}
% this example requires the tcolorbox package
\newcommand{\ifmale}[2]{\ifcsvstrcmp{\gender}{m}{#1}{#2}}

\csvreader[head to column names]{address.csv}{}{%
\begin{tcolorbox}[colframe=DarkGray,colback=White,arc=0mm,width=(\linewidth-2pt)/2,
      equal height group=letter,before=,after=\hfill,fonttitle=\bfseries,
      adjusted title={Letter to \name}]
  \ifcsvstrcmp{\degree}{}{\ifmale{Mr.}{Ms.}}{\degree}~\givenname~\name\\
  \street\\\zip~\location
  \tcblower
  {\itshape Dear \ifmale{Sir}{Madam},}\\
  we are pleased to announce you a bonus value of \bonus\%{}
  which will be delivered to \location\ soon.\\\ldots
\end{tcolorbox}}
\end{dispExample}



\clearpage
\subsection{A Graphical Presentation}%
For this example, we use some artificial statistical data given by a CSV file.

%-- file embedded for simplicity --
\begin{tcbverbatimwrite}{data.csv}
land,group,amount
Bayern,A,1700
Baden-Württemberg,A,2300
Sachsen,B,1520
Thüringen,A,1900
Hessen,B,2100
\end{tcbverbatimwrite}
%-- end embedded file --

\csvlisting{data}

Firstly, we survey the file content using
|\csvautobooktabular|.

\begin{dispExample}
% needs the booktabs package
\csvautobooktabular{data.csv}
\end{dispExample}

The amount values are presented in the following diagram by bars where
the group classification is given using different colors.

\begin{dispExample}
% This example requires the package tikz
\begin{tikzpicture}[Group/A/.style={left color=red!10,right color=red!20},
                    Group/B/.style={left color=blue!10,right color=blue!20}]
\csvreader[head to column names]{data.csv}{}{%
  \begin{scope}[yshift=-\thecsvrow cm]
  \path [draw,Group/\group] (0,-0.45)
    rectangle node[font=\bfseries] {\amount} (\amount/1000,0.45);
  \node[left] at (0,0) {\land};
  \end{scope}  }
\end{tikzpicture}
\end{dispExample}


\clearpage
It would be nice to sort the bars by length, i.\,e.\ to sort the CSV file
by the |amount| column. If the \csvsorter\ program is properly installed,
see Subsection~\ref{sec:Sorting} on page~\pageref{sec:Sorting},
this can be done with the following configuration file for \csvsorter:

\xmllisting{amountsort}

Now, we just have to add an option |sort by=amountsort.xml|:
\begin{dispExample}
% This example requires the package tikz
% Also, the CSV-Sorter tool has to be installed
\begin{tikzpicture}[Group/A/.style={left color=red!10,right color=red!20},
                    Group/B/.style={left color=blue!10,right color=blue!20}]
\csvreader[head to column names,sort by=amountsort.xml]{data.csv}{}{%
  \begin{scope}[yshift=-\thecsvrow cm]
  \path [draw,Group/\group] (0,-0.45)
    rectangle node[font=\bfseries] {\amount} (\amount/1000,0.45);
  \node[left] at (0,0) {\land};
  \end{scope}  }
\end{tikzpicture}
\end{dispExample}




\clearpage
Next, we create a pie chart by calling |\csvreader| twice.
In the first step, the total sum of amounts is computed, and in the second
step the slices are drawn.

\begin{dispExample}
% Modified example from www.texample.net for pie charts
% This example needs the packages tikz, xcolor, calc
\definecolorseries{myseries}{rgb}{step}[rgb]{.95,.85,.55}{.17,.47,.37}
\resetcolorseries{myseries}%

% a pie slice
\newcommand{\slice}[4]{
  \pgfmathsetmacro{\midangle}{0.5*#1+0.5*#2}
  \begin{scope}
    \clip (0,0) -- (#1:1) arc (#1:#2:1) -- cycle;
    \colorlet{SliceColor}{myseries!!+}%
    \fill[inner color=SliceColor!30,outer color=SliceColor!60] (0,0) circle (1cm);
  \end{scope}
  \draw[thick] (0,0) -- (#1:1) arc (#1:#2:1) -- cycle;
  \node[label=\midangle:#4] at (\midangle:1) {};
  \pgfmathsetmacro{\temp}{min((#2-#1-10)/110*(-0.3),0)}
  \pgfmathsetmacro{\innerpos}{max(\temp,-0.5) + 0.8}
  \node at (\midangle:\innerpos) {#3};
}

% sum of amounts
\csvreader[before reading=\def\mysum{0}]{data.csv}{amount=\amount}{%
  \pgfmathsetmacro{\mysum}{\mysum+\amount}%
}

% drawing of the pie chart
\begin{tikzpicture}[scale=3]%
\def\mya{0}\def\myb{0}
\csvreader[head to column names]{data.csv}{}{%
  \let\mya\myb
  \pgfmathsetmacro{\myb}{\myb+\amount}
  \slice{\mya/\mysum*360}{\myb/\mysum*360}{\amount}{\land}
}
\end{tikzpicture}%
\end{dispExample}


\clearpage
Finally, the filter option is demonstrated by separating the groups A and B.
Every item is piled upon the appropriate stack.

\begin{dispExample}
\newcommand{\drawGroup}[2]{%
  \def\mya{0}\def\myb{0}
  \node[below=3mm] at (2.5,0) {\bfseries Group #1};
  \csvreader[head to column names,filter equal={\group}{#1}]{data.csv}{}{%
    \let\mya\myb
    \pgfmathsetmacro{\myb}{\myb+\amount}
    \path[draw,top color=#2!25,bottom color=#2!50]
      (0,\mya/1000) rectangle node{\land\ (\amount)} (5,\myb/1000);
}}

\begin{tikzpicture}
  \fill[gray!75] (-1,0) rectangle (13,-0.1);
  \drawGroup{A}{red}
  \begin{scope}[xshift=7cm]
  \drawGroup{B}{blue}
  \end{scope}
\end{tikzpicture}

\end{dispExample}


\clearpage
\subsection{Macro code inside the data}\label{macrocodexample}%

If needed, the data file may contain macro code. Note that the first character
of a data line is not allowed to be the backslash '|\|'.

%-- file embedded for simplicity --
\begin{tcbverbatimwrite}{macrodata.csv}
type,description,content
M,A nice \textbf{formula},         $\displaystyle \int\frac{1}{x} = \ln|x|+c$
G,A \textcolor{red}{colored} ball, {\tikz \shadedraw [shading=ball] (0,0) circle (.5cm);}
M,\textbf{Another} formula,        $\displaystyle \lim\limits_{n\to\infty} \frac{1}{n}=0$
\end{tcbverbatimwrite}
%-- end embedded file --

\csvlisting{macrodata}

Firstly, we survey the file content using
|\csvautobooktabular|.

\begin{dispExample}
\csvautobooktabular{macrodata.csv}
\end{dispExample}


\begin{dispExample}
\csvstyle{my enumerate}{head to column names,
  before reading=\begin{enumerate},after reading=\end{enumerate}}

\csvreader[my enumerate]{macrodata.csv}{}{%
  \item \description:\par\content}

\bigskip
Now, formulas only:
\csvreader[my enumerate,filter equal={\type}{M}]{macrodata.csv}{}{%
  \item \description:\qquad\content}
\end{dispExample}

\clearpage
\subsection{Tables with Number Formatting}\label{numberformatting}%

We consider a file with numerical data which should be pretty-printed.

%-- file embedded for simplicity --
\begin{tcbverbatimwrite}{data_numbers.csv}
month,    dogs, cats
January,  12.50,12.3e5
February, 3.32, 8.7e3
March,    43,   3.1e6
April,    0.33, 21.2e4
May,      5.12, 3.45e6
June,     6.44, 6.66e6
July,     123.2,7.3e7
August,   12.3, 5.3e4
September,2.3,  4.4e4
October,  6.5,  6.5e6
November, 0.55, 5.5e5
December, 2.2,  3.3e3
\end{tcbverbatimwrite}

\csvlisting{data_numbers}

The |siunitx| package provides a new column type |S|
which can align material using a number of different strategies.
The following example demonstrates the application with CSV reading.
The package documentation of |siunitx| contains a huge amount
of formatting options.

\begin{dispExample}
% \usepackage{siunitx,array,booktabs}
\csvloop{
  file=data_numbers.csv,
  head to column names,
  before reading=\centering\sisetup{table-number-alignment=center},
  tabular={lSS[table-format=2.2e1]@{}c},
  table head=\toprule\textbf{Month} & \textbf{Dogs} & \textbf{Cats} &\\\midrule,
  command=\month & \dogs & \cats &,
  table foot=\bottomrule}
\end{dispExample}

\clearpage
Special care is needed, if the \emph{first} or the \emph{last} column is to be formatted with
the column type |S|. The number detection of |siunitx| is disturbed by
the line reading code of |csvsimple-legacy| which actually is present at the
first and last column. To avoid this problem, the content of the first and last column
could be formatted not by the table format definition, but by using a
suitable |\tablenum| formatting directly, see |siunitx|.

Another and very nifty workaround suggested by Enrico Gregorio is to
add an invisible dummy column with |c@{}| as first column
and |@{}c| as last column:


\begin{dispExample}
% \usepackage{siunitx,array,booktabs}
\csvloop{
  file=data_numbers.csv,
  head to column names,
  before reading=\centering\sisetup{table-number-alignment=center},
  tabular={c@{}S[table-format=2.2e1]S@{}c},
  table head= & \textbf{Cats} & \textbf{Dogs} & \\\midrule,
  command= & \cats & \dogs &,
  table foot=\bottomrule}
\end{dispExample}


\clearpage
Now, the preceding table shall be sorted by the \emph{cats} values.
If the \csvsorter\ program is properly installed,
see Subsection~\ref{sec:Sorting} on page~\pageref{sec:Sorting},
this can be done with the following configuration file for \csvsorter:

\xmllisting{catsort}

Now, we just have to add an option |sort by=catsort.xml|:
\begin{dispExample}
% \usepackage{siunitx,array,booktabs}
% Also, the CSV-Sorter tool has to be installed
\csvloop{
  file=data_numbers.csv,
  sort by=catsort.xml,
  head to column names,
  before reading=\centering\sisetup{table-number-alignment=center},
  tabular={lSS[table-format=2.2e1]@{}c},
  table head=\toprule\textbf{Month} & \textbf{Dogs} & \textbf{Cats} & \\\midrule,
  command=\month & \dogs & \cats &,
  table foot=\bottomrule}
\end{dispExample}


\clearpage
\subsection{CSV data without header line}\label{noheader}%
CSV files with a header line are more semantic than files without header,
but it's no problem to work with headless files.

For this example, we use again some artificial statistical data given by a CSV file
but this time without header.

%-- file embedded for simplicity --
\begin{tcbverbatimwrite}{data_headless.csv}
Bayern,A,1700
Baden-Württemberg,A,2300
Sachsen,B,1520
Thüringen,A,1900
Hessen,B,2100
\end{tcbverbatimwrite}
%-- end embedded file --

\csvlisting{data_headless}

Note that you cannot use the \refKey{/csv/no head} option for the auto tabular
commands. If no options are given, the first line is interpreted as header line
which gives an unpleasant result:

\begin{dispExample}
\csvautobooktabular{data_headless.csv}
\end{dispExample}

To get the expected result, one can redefine \refKey{/csv/table head}
using \refCom{csvlinetotablerow} which holds the first line data for the
|\csvauto...| commands:

\begin{dispExample}
\csvautobooktabular[table head=\toprule\csvlinetotablerow\\]{data_headless.csv}
\end{dispExample}

This example can be extended to insert a table head for this headless data:

\begin{dispExample}
\csvautobooktabular[table head=\toprule\bfseries Land & \bfseries Group
  & \bfseries Amount\\\midrule\csvlinetotablerow\\]{data_headless.csv}
\end{dispExample}

\clearpage

For the normal \refCom{csvreader} command, the \refKey{/csv/no head} option
should be applied. Of course, we cannot use \refKey{/csv/head to column names}
because there is no head, but the columns can be addressed by their numbers:

\begin{dispExample}
\csvreader[no head,
  tabular=lr,
  table head=\toprule\bfseries Land & \bfseries Amount\\\midrule,
  table foot=\bottomrule]
  {data_headless.csv}
  {1=\land,3=\amount}
  {\land & \amount}
\end{dispExample}


\clearpage
\subsection{Imported CSV data}\label{sec:importeddata}%
If data is imported from other applications, there is not always a choice
to format in comma separated values with curly brackets.

Consider the following example data file:

%-- file embedded for simplicity --
\begin{tcbverbatimwrite}{imported.csv}
"name";"address";"email"
"Frank Smith";"Yellow Road 123, Brimblsby";"frank.smith@organization.org"
"Mary May";"Blue Alley 2a, London";"mmay@maybe.uk"
"Hans Meier";"Hauptstraße 32, Berlin";"hans.meier@corporation.de"
\end{tcbverbatimwrite}
%-- end embedded file --

\csvlisting{imported}

If the \csvsorter\ program is properly installed,
see Subsection~\ref{sec:Sorting} on page~\pageref{sec:Sorting},
this can be transformed on-the-fly
with the following configuration file for \csvsorter:

\xmllisting{transform}

Now, we just have to add an option |sort by=transform.xml| to transform
the input data. Here, we actually do not sort.

\begin{dispExample}
% \usepackage{booktabs,array}
% Also, the CSV-Sorter tool has to be installed
\newcommand{\Header}[1]{\normalfont\bfseries #1}

\csvreader[
  sort by=transform.xml,
  tabular=>{\itshape}ll>{\ttfamily}l,
  table head=\toprule\Header{Name} & \Header{Address} & \Header{email}\\\midrule,
  table foot=\bottomrule]
  {imported.csv}{}{\csvlinetotablerow}
\end{dispExample}

The file which is generated on-the-fly and which is actually read by
|csvsimple-legacy| is the following:

\tcbinputlisting{docexample,listing style=tcbdocumentation,fonttitle=\bfseries,
  listing only,listing file=\jobname_sorted._csv}


\clearpage
\subsection{Encoding}\label{encoding}%
If the CSV file has a different encoding than the \LaTeX\ source file,
then special care is needed.

\begin{itemize}
\item The most obvious treatment is to change the encoding of the CSV file
  or the \LaTeX\ source file to match the other one (every good editor
  supports such a conversion). This is the easiest choice, if there a no
  good reasons against such a step. E.g., unfortunately, several tools
  under Windows need the CSV file to be |cp1252| encoded while
  the \LaTeX\ source file may need to be |utf8| encoded.

\item The |inputenc| package allows to switch the encoding inside the
  document, say from |utf8| to |cp1252|. Just be aware that you should only
  use pure ASCII for additional texts inside the switched region.
\begin{dispListing}
% !TeX encoding=UTF-8
% ....
\usepackage[utf8]{inputenc}
% ....
\begin{document}
% ....
\inputencoding{latin1}% only use ASCII from here, e.g. "Uberschrift
\csvreader[%...
  ]{data_cp1252.csv}{%...
  }{% ....
  }
\inputencoding{utf8}
% ....
\end{document}
\end{dispListing}

\item As a variant to the last method, the encoding switch can be done
  using options from |csvsimple-legacy|:
\begin{dispListing}
% !TeX encoding=UTF-8
% ....
\usepackage[utf8]{inputenc}
% ....
\begin{document}
% ....
% only use ASCII from here, e.g. "Uberschrift
\csvreader[%...
  before reading=\inputencoding{latin1},
  after reading=\inputencoding{utf8},
  ]{data_cp1252.csv}{%...
  }{% ....
  }
% ....
\end{document}
\end{dispListing}

\pagebreak\item
If the \csvsorter\ program is properly installed,
see Subsection~\ref{sec:Sorting} on page~\pageref{sec:Sorting},
the CSV file can be re-encoded on-the-fly
with the following configuration file for \csvsorter:

\xmllisting{encoding}

\begin{dispListing}
% !TeX encoding=UTF-8
% ....
\usepackage[utf8]{inputenc}
% ....
\begin{document}
% ....
\csvreader[%...
  sort by=encoding.xml,
  ]{data_cp1252.csv}{%...
  }{% ....
  }
% ....
\end{document}
\end{dispListing}


\end{itemize}


\clearpage

\printindex

\end{document}
